\chapter{Background}
\label{ch:background}

\section{Mesoscale Convective Systems (MCSs)}

\acrfullpl{mcs} constitute a critical component of regional weather forecasting and climatology due to their significant size, duration, and impact. Officially, \acrshortpl{mcs} are defined as a complex of thunderstorms which become organised on a scale larger than any of the individual thunderstorms \citep{NOAANWS2025}. Consequently, these storm systems often last for several hours and cover areas of tens of thousands of square kilometres. In addition, they often produce severe weather phenomena, including flooding, strong winds, and hail \citep{Houze2014}. Unlike many \Gls{synopticscale} systems, \acrshortpl{mcs} are not usually associated with a well-defined centre of circulation and instead are characterised by their multi-scale organisation, typically incorporating a variety of convective cells and larger-scale features, such as squall lines or mesoscale convective complexes \citep{AMS2024,NOAANWS2025}. These systems are prevalent throughout the world and thus are key to understanding regional climatology. For example, in the United States, \acrshortpl{mcs} are a primary driver of warm-season precipitation over the Great Plains \citep{Haberlie2019}. Similarly, the Sahel region of Africa is heavily affected by these storms, producing some of the strongest \acrshortpl{mcs} globally due to it being a climatic transition zone with strong seasonal cycles \citep{Zipser2006}.

\subsection{MCSs in the Horn of Africa}

The Horn of Africa is a region with complex topography and large-scale climatic variability that affects \acrshort{mcs} development. From the west, the Sahel fades into the Ethiopian Highlands and later the East African Rift Valley, characterised by high topographic relief and complex orography. In the east, the Ethiopian Highlands transition into the low-lying coastal plains of Somalia. Unlike most other countries at this latitude, most of Somalia is arid or semi-arid, with the exception of its border region with Kenya \citep{Beck2023}. This contrast in geography is reflected in storm development and precipitation patterns. The mountains of Ethiopia dominate local convective processes \citep{Negash2024} while the low-lying areas on the south-eastern coast of the region are not nearly as conducive to \acrshort{mcs} development and thus are much more susceptible to storm patterns over the Indian Ocean and the Gulf of Aden \citep{Camberlin2024}. The combination of land surface temperature and soil moisture also impact the storm development and intensification. Notably, multiple studies over distinct regions have demonstrated that strong soil moisture gradients can intensify convection \citep{Barton2021,Klein2020,Taylor2017}. These processes are likewise relevant to the Horn of Africa, especially in transitional climates bridging arid and humid zones. Large-scale \glspl{teleconnection} also play a major role in governing \acrshort{mcs} activity in this region. The \acrfull{mjo} is a dominant intra-seasonal factor which modulates rainfall in the tropics and in East Africa, its active phases coincide with increased convection and extreme rainfall events \citep{Camberlin2019,Ochieng2023,Pohl2006}. Quite uniquely for the tropics, the \acrfull{itcz} passes over the region twice per year leading to two distinct rainy seasons, one short and one long \citep{Palmer2023,Tefera2025}. The \acrfull{tej} is another key feature enhancing vertical wind shear which can both facilitate or hinder \acrshort{mcs} development \citep{Farnsworth2011,Vashisht2021}. Additionally, the \acrfull{enso} and the \acrfull{iod} have also been shown to modulate rainfall patterns in the region, with coupled regional climate models able to reproduce these patterns at various timescales \citep{Dubache2019,Endris2019,Vashisht2021,Zaroug2014}. 

Overall, despite this complex array of factors, \acrshortpl{mcs} have been shown to account for over 60\% of extreme rainfall in Ethiopia and Somalia \citep{Hill2023}. These anomalous events often leads to intense flooding, which not only can effect the local population, but also countries as far as Egypt due to the Ethiopian Highlands being a watershed origin for countless rivers including the Blue Nile \citep{Legese2020,Zaroug2014}. Thus, the importance of understanding these storms, predicting their behaviour, and anticipating their effects cannot be overstated. But due to the long history of colonization and underdevelopment, the entire continent of Africa is considerably underrepresented in high-quality meteorological data and many national meteorological services lack critical resources to more effectively monitor and predict \acrshortpl{mcs} \citep{Dinku2019,Kinyondo2018,Meque2021}. This discrepancy also has downstream effects on global meteorology, as reanalysis datasets and digital twin systems rely heavily on data assimilation of observations and regional models for calibration abd parametrisation \citep{Linsenmeier2023,Valmassoi2023}. Although the primary remedy would be increased investment and study, this discrepancy presents an opportunity for otherwise unorthodox methods to make use of existing data to improve predictions and understanding of \acrshortpl{mcs}.

\section{Data-driven Scientific Discovery}

Although scientific research is no stranger to the analysis of experimental data to confirm hypothesis, since the advent of computers in the 20th century, the possibility of data-driven discovery has gained traction. The physical sciences have traditionally favoured numerical methods for complex system modelling, most often relying on the resultant equations from first principles and physical laws as the basis for simulation. Since most of these systems are too complex to be solved analytically, numerical methods are used to approximate solutions within certain constraints limiting grid resolution, stability, and subgrid parametrisation \citep{Lynch2008}. The primary benefit of such techniques is that they are inherently based on well-established physical laws, which can provide a degree of interpretability and trustworthiness. However, these methods famously struggle with the non-linear and chaotic nature of many physical systems, particularly in meteorology where small changes can lead to vastly different outcomes \citep{Lorenz1963}. Other fields, such as biology and social sciences, have favoured statistical approaches with a focus on hypothesis testing, often using techniques such as linear regression, dimensionality reduction, and Bayesian inference to analyse data \insertref{statstical techniques in social sciences and biology}. While these would undoubtedly be considered subcategories of \acrshort{ml}, the tools remain largely supplemental to the dominant research methodologies in these fields.

In contrast, principally data-driven discovery methods have historically been avoided due the lack of data and computational infeasibility, even when such methods were known and their potential well-understood \insertref{EARLY NEURAL NETS}. However, the advent of large datasets and ever-increasing computational power as hardware manufacturers kept pace with Moore's Law \todo{glossary entry} has led to a resurgence of interest in data-driven methods \insertref{MOORES LAW}. Particular credit can be attributed to the surprise success of \acrfull{dl} in computer vision and natural language processing, which has led to a broader acceptance of advanced \acrfull{ml} techniques across many scientific disciplines \insertref{ALEXNET and something NLP}. The key challenge remains the integration of these data-driven methods with existing scientific knowledge and frameworks as this requires not only advances in algorithmic design but also domain expertise. However, it is only a matter of time as more funding and effort are applied. In drug development, \acrfull{ml} appears to be revolutionising the current search for the secrets of protein folding \insertref{alphafold}. \acrshort{dl} has been applied to quantum phase transitions, reaching parity with traditional methods and significantly less compute \insertref{quantum phase transition DL}.

The field of meteorology has been no exception to the increasing popularity of data-driven methods, with a growing body of literature exploring the application of \acrshort{ml} techniques to various meteorological problems. Ample amounts of manual observations spanning nearly centuries, petabytes of satellite imagery since the early 1990s, and the output of numerical models from national forecast offices worldwide, constitute a formidable treasure trove for training and validating these models \citep{Waqas2024}. Given the profound complexity of atmospheric processes, there are numerous opportunities at varying scales for applying \acrshort{ml} to improve our understanding and prediction of weather and climate phenomena. On the smaller scale, \acrshort{ml} has been used for tasks such as downscaling, parameterisation, and nowcasting, where the goal is to improve the resolution and accuracy of existing numerical forecasts \citep{Blunn2024,Zhang2023}. On the larger scale, reanaylsis datasets like the ECMWF's ERA5 and NASA's MERRA-2 have enabled the development of global models that can predict large-scale weather patterns with the partial goal of matching the performance of NWP models \insertref{ERA5, MERRA-2}.

Regardless of the rapid emergence and success of these methods, one fundamental challenge remains: the trustworthiness of \acrshort{ml} models. The black-box nature of many \acrshort{ml} algorithms, particularly \acrshort{dl} networks, poses significant challenges for understanding how these models make predictions and for ensuring that they are robust and reliable \insertref{black box nature of ML}. This is especially important in meteorology, where the consequences of incorrect predictions can be severe. As a result, there is a growing interest in developing methods for explainable and interpretable \acrshort{ml} (XAI) \todo{glossary entry} to provide insights into the decision-making processes of these models \insertref{XAI SOURCE}. As especially popular approach is to incorporate physical constraints and knowledge into the training process often directly in the architecture or loss function, incentivising the models adhere to the laws of physics known to govern atmospheric processes \citep{Dabrowski2020,Chen2022,Luo2025,Zhang2023}. Referred to as PIML \todo{acronym}, this approach has been shown promise in improving model performance and physical accuracy, but concerns still remain regarding interpretability and long-tail events in training data \citep{Sun2025}. Regardless, even while many seasoned meteorologists remain sceptical, it is clear that the integration of \acrshort{ml} into meteorology is a rapidly evolving field with significant potential.

\section{Explainable ML (XAI)}

Explainable AI (XAI) \todo{glossary entry} is an emerging field focused on developing methods and techniques to make the decision-making processes of \acrshort{ml} solutions more transparent and interpretable. This is particularly important in domains where understanding the rationale behind predictions is crucial, such as healthcare, finance, and meteorology. Approaches in this field take a variety of forms, including inherently interpretable models and post-hoc explanation methods. An summary of XAI taxonomy is given in Figure \ref{fig:xai-taxonomy} as conceived by \insertref{XAI book}.

\begin{figure}[h]
    \centering
    \missingfigure{XAI taxonomy figure}
    \caption{XAI Taxonomy.}
    \label{fig:xai-taxonomy}
\end{figure}

Inherently interpretable models are designed to be transparent by construction, often using simpler architectures or feature-based approaches that allow for direct interpretation of the relationship between inputs and outputs. Examples include linear regression, decision trees, and rule-based systems. A clear disadvantage is that these simpler models may sacrifice some predictive performance compared to more complex ones like deep neural networks \todo{acronym for DNNs}. PIML could also be considered a part of this landscape, as it aims to integrate existing physical knowledge into architectures and training processes to enhance forecast accuracy and credibility. Post-hoc explanation methods, on the other hand, are applied to models after training to partially interpret their behaviour. Model-specific post-hoc methods directly leverage the learned structure of the model. For example, great strides have been made in pinning the internal embeddings and attention mechanisms of transformer models to real-world entities and concepts \insertref{embedding map website for claude? there's gotta be papers on this}. Notably for this research, model-agnostic methods are predominantly designed to make no assumptions as to underlying model architecture, making them some of the most generally applicable and widely used techniques in XAI \insertref{stats for XAI library usage}. These methods can be broadly categorised into local and global explanations. Local explanations focus on individual predictions, providing insights into model behaviour for a given input. In contrast, global explanations aim to provide an overall understanding of the model's behaviour for any input. For local explanations, counterfactual analysis has gained traction, especially for classification tasks, where the model's local decision boundary can be more interactively explored through slight perturbations of the a given input \citep{Mothilal2019}. Techniques such as LIME (Local Interpretable Model-agnostic Explanations) and SHAP (SHapley Additive exPlanations) are commonly used for both objectives \citep{Lundberg2017}\insertref{lime}. Though, LIME does have limited advantages over SHAP in specific use cases, \cite{Lundberg2017} showed it to be a subset of SHAP and thus a brief overview of SHAP follows.

\subsection{SHapley Additive exPlanations (SHAP)}

After its introduction in \cite{Lundberg2017}, SHapley Additive exPlanations (SHAP) have emerged as one of the most popular frameworks for post-hoc model explanations. This is achieved through the application of cooperative game theory, specifically the concept of the Shapley value, which fairly distributes the "payout" (i.e., the model's prediction) among the features based on their individual contributions \insertref{original shapley paper}.

\explainindetail{need to talk about coalitions, then the rough outline of how Shapley values are defined theoretically, then maybe how SHAP library from lundberg and lee implements this in practice}

\section{Literature Review}

\subsection{XAI in Meteorology}


For example, LIME has helped identify crucial features in random forest models for seasonal precipitation prediction, with explanations aligning closely to known meteorological phenomena like ENSO impacts[https://www.sciencedirect.com/science/article/abs/pii/S1352231024004722].


\subsection{XAI for Convective Systems}

Case Studies in Convective Systems:

Tropical Convection and MCSs: LRP has revealed the importance of large-scale vertical velocity and wind shear in predicting convective area and organization in the tropics, with moisture and thermodynamic factors primarily affecting the convective area, and horizontal wind fields impacting system organization[https://www.sciencedirect.com/science/article/abs/pii/S1352231024004722].

Paleoclimate reconstruction of Indian monsoon [https://cp.copernicus.org/articles/21/1/2025/].

Relevant Example: Indian Monsoon Lows, Hunt and Turner (2023) demonstrated the utility of interpretable gradient-boosted decision-tree ensembles in uncovering new dynamical relationships governing Indian monsoon low-pressure systems \citep{Hunt2024}. 

Limitations include for Hunt Turner 2023: 
- only applied to indian monsoon. would be interesting to apply to other regions and storm types
- less focus on any prediction capability of models, more just focused on finding novel meteorological relationships

relevant and motivate the work