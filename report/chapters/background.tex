\chapter{Background and Literature Review}
\label{ch:background}

\section{Mesoscale Convective Systems (MCSs)}

\acrfullpl{mcs} constitute a critical component of regional weather forecasting and climatology due to their significant size, duration, and impact. Officially, \acrshortpl{mcs} are defined as a complex of thunderstorms which become organised on a scale larger than any of the individual thunderstorms \citep{NOAANWS2025}. Consequently, these storm systems often last for several hours and cover areas of tens of thousands of square kilometres. In addition, they often produce severe weather phenomena, including flooding, strong winds, and hail \citep{Houze2014}. Unlike many \Gls{synopticscale} systems, \acrshortpl{mcs} are not usually associated with a well-defined centre of circulation and instead are characterised by their multi-scale organisation, typically incorporating a variety of convective cells and larger-scale features, such as squall lines or mesoscale convective complexes \citep{AMS2024,NOAANWS2025}. These systems are prevalent throughout the world and thus are key to understanding regional climatology. For example, in the United States, \acrshortpl{mcs} are a primary driver of warm-season precipitation over the Great Plains \citep{Haberlie2019}. Similarly, the Sahel region of Africa is heavily affected by these storms, producing some of the strongest \acrshortpl{mcs} globally due to it being a climatic transition zone with strong seasonal cycles \citep{Zipser2006}.

\subsection{MCSs in the Horn of Africa}

The Horn of Africa is a region with complex topography and large-scale climatic variability that affects \acrshort{mcs} development. From the west, the Sahel fades into the Ethiopian Highlands and later the East African Rift Valley, characterised by high topographic relief and complex orography. In the east, the Ethiopian Highlands transition into the low-lying coastal plains of Somalia. Unlike most other countries at this latitude, most of Somalia is arid or semi-arid, with the exception of its border region with Kenya \citep{Beck2023}. This contrast in geography is reflected in storm development and precipitation patterns. The mountains of Ethiopia dominate local convective processes \citep{Negash2024} while the low-lying areas on the south-eastern coast of the region are not nearly as conducive to \acrshort{mcs} development and thus are much more susceptible to storm patterns over the Indian Ocean and the Gulf of Aden \citep{Camberlin2024}. The combination of land surface temperature and soil moisture also impact the storm development and intensification. Notably, multiple studies over distinct regions have demonstrated that strong soil moisture gradients can intensify convection \citep{Barton2021,Klein2020,Taylor2017}. These processes are likewise relevant to the Horn of Africa, especially in transitional climates bridging arid and humid zones. Large-scale \glspl{teleconnection} also play a major role in governing \acrshort{mcs} activity in this region. The \acrfull{mjo} is a dominant intra-seasonal factor which modulates rainfall in the tropics and in East Africa, its active phases coincide with increased convection and extreme rainfall events \citep{Ochieng2023,Pohl2006}. Quite uniquely for the tropics, the \acrfull{itcz} passes over the region twice per year leading to two distinct rainy seasons, one short and one long \citep{Palmer2023,Tefera2025}. The \acrfull{enso} and the \acrfull{iod} have also been shown to modulate rainfall patterns in the region, with coupled regional climate models able to reproduce these patterns at various timescales \citep{Dubache2019,Endris2019,Vashisht2021}. Overall, despite this complex array of factors, \acrshortpl{mcs} have been shown to account for over 60\% of extreme rainfall in Ethiopia and Somalia \citep{Hill2023}.

