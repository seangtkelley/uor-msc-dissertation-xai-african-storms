\chapter{Background and Literature Review}
\label{ch:background}

% A literature review chapter can be organized in a few sections with appropriate titles. A literature review chapter might  contain the following:
% \begin{enumerate}
%     \item A review of the state-of-the-art (include theories and solutions) of the field of research.
%     \item A description of the project in the context of existing literature and products/systems.
%     \item An analysis of how the review is relevant to the intended application/system/problem.
%     \item A critique of existing work compared with the intended work.
% \end{enumerate}
% Note that your literature review should demonstrate the significance of the project.
% https://guides.library.bloomu.edu/litreview

% \section{State-of-the-art}

% \section{Critique of the review}

% \section{Summary} 

\section{Mesoscale Convective Systems (MCSs)}

Due to significant size, duration, and impact, \acrfullpl{mcs} constitute a critical component of regional weather forecasting and climatology. Officially, \acrshortpl{mcs} are defined as complex of thunderstorms which becomes organised on a scale larger than any of the individual thunderstorms \citep{NOAANWS2025}. Consequently, these storm systems often last for several hours and cover areas of tens of thousands of square kilometres. In addition, they often produce severe weather phenomena, including flooding, strong winds, and hail \citep{Houze2014}. Unlike many \Gls{synopticscale} systems, \acrshortpl{mcs} are not usually associated with a well-defined center of circulation and instead are characterized by their multi-scale organization, typically incorporating a variety of convective cells and larger-scale features such as squall lines or mesoscale convective complexes \citep{NOAANWS2025,AMS2024}. These systems are prevalent throughout the world and thus are key to understanding regional climatology. For example, in the United States, \acrshortpl{mcs} are a primary driver of warm-season precipitation over the Great Plains \citep{Haberlie2019}. The Sahel region of Africa produces some of the strongest \acrshortpl{mcs} globally due to it being a climatic transition zone with strong seasonal cycles \citep{Zipser2006}.

\subsection{MCSs in the Horn of Africa}

While the Horn of Africa is a region with complex topography and large-scale climatic variability affecting the development of \acrshortpl{mcs}, it has been shown that the systems still contribute to over 60\% of extreme rainfall in Ethiopia and Somalia \citep{Hill2023}. From the west, the Sahel fades into the Ethiopian Highlands and later the East African Rift Valley, characterized by high topographic relief and complex orography which dominate convective processes in the region \citep{Negash2024}. In the east, the Ethiopian Highlands transition into the low-lying coastal plains of Somalia. Unlike most other countries at this latitude, most of Somalia is arid or semi-arid, with the exception of its border region with Kenya \citep{Beck2023}.

 Proximity to both the Red Sea and Indian Ocean influences surface humidity and temperature, providing moisture sources and controlling regional monsoon dynamics. Sea surface temperature (SST) anomalies can have teleconnected impacts on rainfall extremes through the modulation of atmospheric circulation[3]. The pattern of land surface temperature and soil moisture, largely dictated by vegetation and recent precipitation, impacts the development and intensification of MCSs. Notably, studies in the Sahel have demonstrated that dry soils downstream of moisture anomalies can intensify convection by triggering enhanced surface sensible heat flux, increasing instability, and strengthening low-to-mid-level wind shear[4][5]. These processes are similarly relevant to the Horn of Africa, especially in transitional climates bridging arid and humid zones.

Large-scale scale teleconnections also play a major role in governing regional climate patterns and MCS activity. The EAJ, along with the African Easterly Jet (AEJ), is critical in providing the wind shear required for the growth and westward propagation of MCSs. Enhanced meridional temperature gradients, often influenced by soil moisture and orography, strengthen these jets and facilitate system longevity[6][2]. The MJO is a dominant intraseasonal oscillation, modulating rainfall variability in the tropics[7][8][9]. In East Africa, its active phases coincide with increased convection and extreme rainfall events, partly due to anomalous low-level westerlies fostering moisture advection and instability that fuel MCS development. El Niño-Southern Oscillation (ENSO) and the Indian Ocean Dipole (IOD) have also been shown to modulate rainfall patterns in the region, with coupled regional climate models able to reproduce these patterns at various timescales[3][10].

