\chapter{Discussion}
\label{ch:discuss}
\todo{in progress}

This chapter discusses the implications of the findings, the limitations of the study, and potential directions for future research.

\section{Model Performance and Predictability}

\subsection{First Points Only vs All Points}

Models trained on only the first observation in each storm consistently underperform compared to those trained on all available observations. This suggests that incorporating the full temporal evolution of storms provides valuable information for prediction.

\subsection{Storm Direction and Distance}

As shown in Tables \ref{tab:storm_direction_results} and \ref{tab:obs_experiment_results} respectively, storm direction models perform well, but models predicting the next direction and distance of the storm at an observation perform poorly. This could be due to several factors. The calculation of the next direction relies on the centroid of the storm provided by \cite{Hill2023}. For \acrshortpl{mcs}, the centroid might be weakly defined as the storm structure does not often have a clear centre of circulation compared to other systems like tropical cyclones. For example, squall lines are linear features and the centroid location outputted by the tracking algorithm may vary significantly between each time step based on the shape and orientation of the line. This would lead to particularly noisy data. To mitigate this, future work might investigate smoothing the centroid trajectory to reduce noise and improve the reliability of movement estimates. For direction specifically, using categorical representations of the compass bearing or transforming the degrees using sine and cosine functions, could address the challenges of modelling circular data. Categorical approaches would be more intuitive, but sine and cosine transformations would preserve granularity at the cost of increased complexity and the need for multiple outputs.

\section{Future Work}

\subsection{Dataset Expansion and Feature Engineering}

The \acrshort{mcs} database used in this study was limited to storms which spent their entire lifecycle within the study region. Expanding the storm database to include storms that pass through the region, rather than only those that originate and terminate within it, might provide a more comprehensive dataset for analysis. 

Additional features, such as equivalent potential temperature, could be incorporated to enhance model inputs.

The use of supplementary data sources, including satellite imagery and precipitation data as demonstrated in the original paper \citep{Hill2023}, may further improve predictive capabilities. While the universal availability of \acrshort{era5} data makes it a more practical choice for widespread application, region-specific data or higher resolution products could offer enhanced detail and accuracy in certain contexts even if they are less accessible or reduce viable data coverage. For example, this thesis considered the inclusion of precipitation data from the Global Precipitation Measurement (GPM) mission, but ultimately did not include it due to over 70\% of storm observations lacking corresponding GPM data. Future work could explore methods to effectively integrate such datasets.

\subsection{Model Improvements and Explainability}

\begin{itemize}
    \item Explore the use of alternative machine learning algorithms and architectures to improve predictive performance.
    \begin{itemize}
        \item Investigating anomalous individual predictions using decision plots or force plots to quantify edge cases for the models.
    \end{itemize}
    \item Conducting further analysis on the interpretability of model predictions, potentially through the use of more advanced explainability techniques.
    \begin{itemize}
        \item Investigating anomalous individual predictions using decision plots or force plots to quantify edge cases for the models.
    \end{itemize}
\end{itemize}