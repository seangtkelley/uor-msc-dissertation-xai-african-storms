\newglossaryentry{machinelearning}
{
    name=Machine Learning,
    description={Subset of artificial intelligence that enables systems to learn from data and improve their performance over time without being explicitly programmed}
}
\newglossaryentry{artificialneuralnetwork}
{
    name=Artificial Neural Network,
    description={Computational model consisting of interconnected nodes (neurons) that process information in layers, inspired by the structure and function of biological neural networks}
}
\newglossaryentry{deeplearning}
{
    name=Deep Learning,
    description={Subset of machine learning that leverages \acrfull{ann} with many layers to learn complex, non-linear patterns from large amounts of data}
}
\newglossaryentry{deepneuralnetwork}
{
    name=Deep Neural Network,
    description={Type of \acrfull{ann} used in \acrfull{dl}}
}
\newglossaryentry{mesoscale}
{
    name=Mesoscale,
    description={Meteorological phenomena that occur at a horizontal scale of 5 to 200 kilometres}
}
\newglossaryentry{mesoscaleconvectivesystem}
{
    name=Mesoscale Convective System,
    description={A group of thunderstorms organised into a single cloud system that lasts several hours, often resulting in extreme rainfall, flash flooding and hail}
}
\newglossaryentry{synopticscale}
{
    name=Synoptic Scale,
    description={Meteorological phenomena that occur at a horizontal scale of 200 kilometres and above}
}
\newglossaryentry{teleconnection}
{
    name=Teleconnection,
    text={teleconnection},
    description={Climate patterns related to each other at large distances, typically thousands of kilometres apart}
}
\newglossaryentry{maddenjulianoscillation}
{
    name=Madden-Julian Oscillation,
    description={Main component of tropical intraseasonal variability via a coupling of circulation and convection that travels slowly eastward over the Indian and Pacific Oceans}
}
\newglossaryentry{elninosouthernoscillation}
{
    name=El Niño-Southern Oscillation,
    description={A shift in position of sea surface pressure anomalies between each side of the tropical Pacific Ocean with a period of 2-5 years}
}
\newglossaryentry{indianoceandipole}
{
    name=Indian Ocean Dipole,
    description={An irregular oscillation of sea surface temperatures in the Indian Ocean}
}
\newglossaryentry{intertropicalconvergencezone}
{
    name=Intertropical Convergence Zone,
    description={A belt near the equator where the northeasterly and southeasterly trade winds converge}
}
\newglossaryentry{tropicaleasterlyjet}
{
    name=Tropical Easterly Jet,
    description={A high-altitude, easterly wind current stretching over the tropics from South Asia to Africa which is most prominent during the Asian monsoon}
}
\newglossaryentry{blackbox}
{
    name=Black Box,
    text={black box},
    description={A system or model whose internal workings are not visible or easily understood, with only inputs and outputs accessible for analysis}
}
\newglossaryentry{artint}
{
    name=Artificial Intelligence,
    description={Field of computer science focused on creating systems capable of performing tasks that typically require human intelligence}
}

\newglossaryentry{explai}
{
    name=Explainable Artificial Intelligence,
    description={Methods and techniques in \acrfull{ai} that aim to make the results and workings of models understandable to humans}
}
\newglossaryentry{mooreslaw}
{
    name=Moore's Law,
    description={Observation and prediction named after semiconductor pioneer Gordon Moore that the number of transistors on a microchip doubles approximately every two years}
}
\newglossaryentry{combinatorialexplosion}
{
    name=Combinatorial Explosion,
    description={Rapid growth of algorithmic complexity due to its combinatoric dependence on inputs}
}
\newglossaryentry{lowpressuresystem}
{
    name=Low Pressure System,
    description={A \gls{synopticscale} region where atmospheric pressure is lower than that of surrounding areas, often associated with cloudy weather, precipitation, and increased wind}
}
\newglossaryentry{indianmonsoon}
{
    name=Indian Monsoon,
    description={A seasonal wind pattern over the Indian subcontinent, characterised by heavy rainfall during the summer months due to the reversal of winds and moisture transport from the ocean to land}
}
\newglossaryentry{ensemblelearning}
{
    name=Ensemble Learning,
    text={ensemble learning},
    description={A machine learning paradigm where multiple models are trained to solve the same problem or subdivisions of the same problem and combined to improve overall performance}
}

\newacronym{ml}{ML}{\Gls{machinelearning}}
\newacronym{dl}{DL}{\Gls{deeplearning}}
\newacronym{ann}{ANN}{\Gls{artificialneuralnetwork}}
\newacronym{dnn}{DNN}{\Gls{deepneuralnetwork}}
\newacronym{mcs}{MCS}{\Gls{mesoscaleconvectivesystem}}
\newacronym{mjo}{MJO}{\Gls{maddenjulianoscillation}}
\newacronym{enso}{ENSO}{\Gls{elninosouthernoscillation}}
\newacronym{iod}{IOD}{\Gls{indianoceandipole}}
\newacronym{itcz}{ITCZ}{\Gls{intertropicalconvergencezone}}
\newacronym{tej}{TEJ}{\Gls{tropicaleasterlyjet}}
\newacronym{ecmwf}{ECMWF}{European Centre for Medium-Range Weather Forecasts}
\newacronym{era5}{ERA5}{ECMWF Reanalysis v5}
\newacronym{nasa}{NASA}{National Aeronautics and Space Administration}
\newacronym{merra2}{MERRA-2}{Modern-Era Retrospective Analysis for Research and Applications, Version 2}
\newacronym{ai}{AI}{\Gls{artint}}
\newacronym{xai}{XAI}{\Gls{explai}}
\newacronym{piml}{PIML}{Physics-informed \Gls{machinelearning}}
\newacronym{lime}{LIME}{Local Interpretable Model-agnostic Explanations}
\newacronym{shap}{SHAP}{SHapley Additive exPlanations}
\newacronym{lps}{LPS}{Low Pressure System}
\newacronym{nwp}{NWP}{Numerical Weather Prediction}
\newacronym{lrp}{LRP}{Layer-wise Relevance Propagation}
\newacronym{xgb}{XGBoost}{eXtreme Gradient Boosting}
\newacronym{wandb}{W\&B}{\href{https://wandb.ai/}{Weights \& Biases}}
\newacronym{rmse}{RMSE}{Root Mean Square Error}
\newacronym{nan}{NaN}{Not a number}
\newacronym{tcwv}{TCWV}{Total Column Water Vapour}
\newacronym{cape}{CAPE}{Convective Available Potential Energy}
\newacronym{olr}{OLR}{Outgoing Longwave Radiation}