\chapter{Introduction}
\label{ch:intro}

\cite{Gebrechorkos2019}

% \textbf{Guidance on introduction chapter writing:} Introductions are written in the following parts:
% \begin{itemize}
%     \item A brief  description of the investigated problem.
%     \item A summary of the scope and context of the project, i.e., what is the background of the topic/problem/application/system/algorithm/experiment/research question/hypothesis/etc. under investigation/implementation/development [whichever is applicable to your project].
%     \item The aims and objectives of the project.
%     \item A description of the problem and the methodological approach adopted to solve the problem.
%     \item A summary of the most significant outcomes and their interpretations.
%     \item Organization of the report. 
% \end{itemize}


\section{Background}
\label{sec:into_back}
Describe to a reader the context of your project. That is, what is your project and what its motivation. Briefly explain the major theories, applications, and/or products/systems/algorithms whichever is relevant to your project.

\textbf{Cautions:} Do not say you choose this project because of your interest, or your supervisor proposed/suggested this project, or you were assigned this project as your final year project. This all may be true, but it is not meant to be written here.

\section{Problem statement}
\label{sec:intro_prob_art}
This section describes the investigated problem in detail. You can also have a separate chapter on ``Problem articulation.''  For some projects, you may have a section like ``Research question(s)'' or ``Research Hypothesis'' instead of a section on ``Problem statement.'

\section{Aims and objectives}
\label{sec:intro_aims_obj}
Describe the ``aims and objectives'' of your project. 

\textbf{Aims:} The aims tell a reader what you want/hope to achieve at the end of the project. The  aims define your intent/purpose in general terms.  

\textbf{Objectives:} The objectives are a set of tasks you would perform in order to achieve the defined aims. The objective statements have to be specific and measurable through the results and outcome of the project.

\section{Solution approach}
\label{sec:intro_sol}

\section{Summary of contributions and achievements}
\label{sec:intro_sum_results}
Describe clearly what you have done/created/achieved and what the major results and their implications are. 


\section{Report Structure}
\label{sec:intro_org}
Describe the outline of the rest of the report here. Let the reader know what to expect ahead in the report. Describe how you have organized your report. 

\textbf{Example: how to refer a chapter, section, subsection}. This report is organised into seven chapters. Chapter~\ref{ch:lit_rev} details the literature review of this project. In Section~\ref{ch:method}...  % and so on.

\textbf{Note:}  Take care of the word like ``Chapter,'' ``Section,'' ``Figure'' etc. before the \LaTeX~command \textbackslash ref\{\}. Otherwise, a  sentence will be confusing. For example, In \ref{ch:lit_rev} literature review is described. In this sentence, the word ``Chapter'' is missing. Therefore, a reader would not know whether 2 is for a Chapter or a Section or a Figure.  For more information on \textbf{automated tools} to assist in this work, see \Cref{subsec:reftools}.

