\chapter{Conclusion}
\label{ch:con}

Globally, Mesoscale Convective Systems (MCSs) regularly cause flooding and wind damage due to their associated intense rainfall and severe weather. While these systems are signficant in size, the convective processes that drive their intensification and propagation have historically been parametrised in Numerical Weather Prediction (NWP) models
% more here

As a contribution to the work to better understand \acrshortpl{mcs}, this thesis applied a reusable framework, adapted from \cite{Hunt2024}, for predicting their behaviour using \acrfull{xai} techniques. Prerequisite to the approach, a dataset was created that captures relevant meteorological, geographic, and temporal features along the 27,982 storm tracks within the Horn of Africa selected from \cite{Hill2023}. Multiple \acrfull{xgb} models were then trained to predict specific variables as proxies for storm intensification and propagation. The models were optimised using hyperparameter tuning and their performance compared between similar experiments with different feature sets to assess the impact of feature selection. Finally, \acrshort{xai} techniques, particularly \acrshort{shap}, were be used to interrogate the models' predictions and identify the key factors they identified which influence storm behaviour.

% reword this: direction from abstract
The explainability analysis reveals that the machine learning (ML) models predominantly learned to utilise the meteorological variables in a manner that aligns with established literature on storm development. However, open questions remain regarding potential spurious correlations between key regional geographic and meteorological features within the dataset. Models trained to predict propagation are not sufficiently performant to warrant analysis of their learned behaviour. It is also shown that models trained on all available storm observations outperform those trained only on the first observation, highlighting that knowledge of MCS evolution improves predictive skill. This work confirms the potential of XAI techniques to provide insights into the dynamics of MCSs and potentially inform the future development of forecast systems in data-sparse regions. Additionally, through this research and its antecedents, it is evidenced that the methodology is generalisable across different geographies and storm types. 

\section{Future Work}
\label{sec:future-work}

\subsection{Deepening of Explainability Analysis}

Due to the limited time available for this thesis, the explainability analysis conducted was far from exhaustive. The approach of identifying candidate features for further geographic and temporal analysis using their correlation with latitude, longitude, and time was a rough heuristic which may have missed more subtle relationships within feature contributions. Future work should aim to develop more sophisticated methods or more exhaustively examine all top features if time permits.

Furthermore, the analysis primarily focused on global feature importance and summary plots. Future work could delve deeper into local explanations to understand individual predictions better. Initial candidate samples could be easily identified via predictions with anomalous \acrshort{rmse}. Special attention should be paid to instances where similar samples exist that were correctly predicted. To facilitate investigation, the SHAP library provides various visualisation tools, such as force plots and decision plots, which can be utilised to visualise which features contribute to the erroneous output. Such an approach could help identify specific conditions or feature interactions that lead to model errors, thereby guiding targeted model improvements or possibly uncovering edge-case MCS behaviour.

\subsection{Dataset Expansion and Feature Engineering}

The \acrshort{mcs} database used in this study was limited to storms which spent their entire lifecycle within the study region. Expanding the storm database to also include storms tracks that pass through the region might provide a more comprehensive dataset for analysis, especially on the edges of the domain. The domain itself might also be expanded or contracted to reduce geographic correlations apparent with some of the meteorological features. For example, expansion of the dataset east to include more of the Indian Ocean or southwest to include the continuation of the Ethiopia Highlands into Kenya could improve the coverage of oceanic \acrshortpl{mcs} and reduce spatial biases with orography respectively. A reduction of the domain, perhaps to solely capture \acrshortpl{mcs} over the Ethiopian Highlands, could also help eliminate these correlations.

The dataset could also be expanded with additional features, such as equivalent potential temperature. As displayed in table \ref{tab:era5-file-patterns}, equivalent potential temperature data was available from the \acrshort{era5} data, but it was not included due to time constraints. Conversely, some features might be removed, particularly ones which contribute little to model performance, to reduce overall dimensionality and possibly improve model generalisation.

\subsection{Modelling Improvements}

Given this study focuses on post-hoc, model-agnostic explainability methods, future work could explore the integration of more complex \acrshort{ml} models, such as \acrfullpl{dnn}, to potentially improve predictive performance. While these models are not inherently interpretable, the methodology of applying of SHAP to explain their predictions would remain mostly unchanged when compared this thesis. Alternatively, more advanced, model-specific explainability techniques, such as layer-wise relevance propagation, or partially interpretable models, like attention-based \acrshortpl{dnn}, could be implemented.