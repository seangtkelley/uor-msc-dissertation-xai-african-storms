\chapter{Conclusion}
\label{ch:con}

Globally, Mesoscale Convective Systems (MCSs) regularly cause flooding and wind damage due to their associated intense rainfall and severe weather. While these systems are signficant in size, the convective processes that drive their intensification and propagation have historically been parametrised in Numerical Weather Prediction (NWP) models
% more here

As a contribution to the work to better understand \acrshortpl{mcs}, this thesis applied a reusable framework, adapted from \cite{Hunt2024}, for predicting their behaviour using \acrfull{xai} techniques. Prerequisite to the approach, a dataset was created that captures relevant meteorological, geographic, and temporal features along the 27,982 storm tracks within the Horn of Africa selected from \cite{Hill2023}. Multiple \acrfull{xgb} models were then trained to predict specific variables as proxies for storm intensification and propagation. The models were optimised using hyperparameter tuning and their performance compared between similar experiments with different feature sets to assess the impact of feature selection. Finally, \acrshort{xai} techniques, particularly \acrshort{shap}, were be used to interrogate the models' predictions and identify the key factors they identified which influence storm behaviour.

% reword this: direction from abstract
The explainability analysis reveals that the machine learning (ML) models predominantly learned to utilise the meteorological variables in a manner that aligns with established literature on storm development. However, open questions remain regarding potential spurious correlations between key regional geographic and meteorological features within the dataset. Models trained to predict propagation are not sufficiently performant to warrant analysis of their learned behaviour. It is also shown that models trained on all available storm observations outperform those trained only on the first observation, highlighting that knowledge of MCS evolution improves predictive skill. This work confirms the potential of XAI techniques to provide insights into the dynamics of MCSs and potentially inform the future development of forecast systems in data-sparse regions. Additionally, through this research and its antecedents, it is evidenced that the methodology is generalisable across different geographies and storm types. 

