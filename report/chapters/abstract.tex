\chapter*{\center \Large  Abstract}

\noindent
Mesoscale Convective Systems (MCSs) pose a serious risk to life and property in East Africa due to their intense rainfall and severe weather. Predicting their behaviour remains a significant challenge because their convective processes occur on scales not traditionally resolved by Numerical Weather Prediction (NWP) models. In lieu of direct simulation, NWP models rely on parametrisation schemes to represent sub-grid scale processes, comprising a major source of uncertainty. Newer convection-permitting models are showing promise in more skilfully reproducing complex dynamics. However, large discrepancies still exist, and resources remain limited to deploy and verify such models, especially in Africa. These challenges present an opportunity for innovative methods to identify novel predictive characteristics of MCSs, subsequently guiding further NWP development. In this thesis, Explainable Artificial Intelligence (XAI) techniques are applied to a dataset of tracked MCSs over the Horn of Africa with relevant meteorological, geographic, and temporal variables to identify key features influencing storm intensification and propagation. XGBoost models are first trained to predict proxy variables for storm maximum intensity, rate of intensification, direction, and speed of propagation. Shapley Additive Explanations (SHAP) is then used to inspect their behaviour. The explainability analysis reveals that the machine learning (ML) models predominantly learned to utilise the meteorological variables in a manner that aligns with established literature on storm development. However, open questions remain regarding potential spurious correlations between key regional geographic and meteorological features within the dataset. Models trained to predict propagation are not sufficiently performant to warrant analysis of their learned behaviour. It is also shown that models trained on all available storm observations outperform those trained only on the first observation across all experiments, highlighting that knowledge of MCS evolution improves predictive skill. This work confirms the potential of XAI techniques to provide insights into the dynamics of MCSs and potentially inform the future development of forecast systems in data-sparse regions. Additionally, through this research and its antecedents, it is evidenced that the methodology is generalisable across different geographies and storm types. Future work using this framework for East Africa should aim to deepen the explainability analysis to further understand feature interactions and possibly uncover any learned model behaviour that is not currently catalogued regarding MCSs in the region. For instance, this thesis can help in the parametrisation schemes of mesoscale convection by identifying threshold values or temporal dependencies which can be used to adjust trigger points or feature weighting within said schemes.

~\\[1cm]
\noindent
\textbf{Keywords:} Mesoscale Convective Systems, East Africa, Explainable Artificial Intelligence, XGBoost, SHAP

\vfill
\noindent
\textbf{Word count:} 5119
 \newline
\newline
\noindent
\textbf{Report code:} \href{https://github.com/seangtkelley/uor-msc-dissertation-xai-african-storms}{https://github.com/seangtkelley/uor-msc-dissertation-xai-african-storms}  \newline

