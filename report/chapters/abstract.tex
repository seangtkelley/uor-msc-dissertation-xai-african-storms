\chapter*{\center \Large  Abstract}

\noindent
Mesoscale Convective Systems (MCSs) pose a serious risk to life and property in East Africa due to their associated intense rainfall and severe weather. Predicting their behaviour remains a significant challenge because their convective processes occur on scales not traditionally resolved by Numerical Weather Prediction (NWP) models. In lieu of direct simulation, NWP models rely on parametrisation schemes to represent sub-grid scale processes, but these schemes are a major source of uncertainty. Newer convection-permitting models are showing promise in more skilfully reproducing convective processes, but large discrepancies still exist, and resources remain limited to deploy and verify such models, especially in Africa. These challenges present an opportunity for innovative methods to identify novel predictive characteristics of MCSs, subsequently guiding further NWP development. In this thesis, explainable artificial intelligence (XAI) techniques are applied to MCS data over the Horn of Africa to identify key features influencing storm intensification and propagation. Explainability analysis of XGBoost models trained to predict the maximum intensity of storms and the rate of intensification reveals the models learned to utilise the meteorological variables in a manner that aligns with established literature on storm development. Models trained to predict propagation are not sufficiently performant to warrant further analysis of their learned behaviour. It is also shown that models trained on all available storm observations outperform those trained only on the first observation, highlighting that knowledge of MCS evolution improves predictive skill. This work confirms the potential of XAI techniques to provide insights into the dynamics of MCSs and potentially inform the future development of forecast systems in data-sparse regions. Additionally, through this research and its antecedents, it is evidenced that the methodology is generalisable across different geographies and storm types. Future work using this framework for East Africa might aim to deepen the explainability analysis to uncover any learned model behaviour that is not currently catalogued or understood regarding MCSs in the region. Such lines of investigation could be guided by the current limitations of the most commonly utilised parametrisation schemes.

~\\[1cm]
\noindent
\textbf{Keywords:} Mesoscale Convective Systems, East Africa, Explainable Artificial Intelligence, XGBoost, SHAP

\vfill
\noindent
\textbf{Word count:} \input{wordcount.tex} \newline
\newline
\noindent
\textbf{Report code:} \href{https://github.com/seangtkelley/uor-msc-dissertation-xai-african-storms}{https://github.com/seangtkelley/uor-msc-dissertation-xai-african-storms}  \newline

