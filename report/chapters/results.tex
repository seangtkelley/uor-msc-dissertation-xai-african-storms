\chapter{Results}
\label{ch:results}

The results of the experiments conducted in this study are presented in this chapter. The findings are organized according to the research questions outlined in Chapter \ref{ch:intro}. Each section provides a detailed analysis of the results, including relevant figures and tables to support the findings. Detailed analysis of the results will following in Chapter \ref{ch:discuss}.

\section{Experimental Setup}

For each experiment, the model was trained twice: once using all available data and once using only ERA5 meteorological data. This approach allows for a comparison of model performance when utilizing comprehensive data versus a more focused subset of meteorological features.

\section{Predict Storm Aggregate Features from First Observation}

These experiments aim to predict the overall characteristics of a storm based on its initial observation.

\subsection{Storm Max Intensity}

\begin{figure}[h]
    \centering
    \missingfigure{2x2 Subplots: Row 1: Actual vs Predicted scatter with regression, Row 2: SHAP beeswarm top features.}
    \caption{Comparison of performance and top features for storm max intensity.}
    \label{fig:storm_max_intensity_summary}
\end{figure}

\subsection{Storm Direction}

\begin{figure}[h]
    \centering
    \missingfigure{2x2 Subplots: Row 1: Actual vs Predicted scatter with regression, Row 2: SHAP beeswarm top features.}
    \caption{Comparison of performance and top features for storm direction.}
    \label{fig:storm_direction_summary}
\end{figure}

\subsection{Predict Immediate Characteristics at an Observation}

These experiments aim to predict the immediate characteristics of a storm based on its current observation.

\subsection{Intensification}

\begin{figure}[h]
    \centering
    \missingfigure{2x2 Subplots: Row 1: Actual vs Predicted scatter with regression, Row 2: SHAP beeswarm top features.}
    \caption{Comparison of performance and top features for intensification.}
    \label{fig:obs_intensification_summary}
\end{figure}

\subsection{Direction}

\begin{figure}[h]
    \centering
    \missingfigure{2x2 Subplots: Row 1: Actual vs Predicted scatter with regression, Row 2: SHAP beeswarm top features.}
    \caption{Comparison of performance and top features for direction.}
    \label{fig:obs_direction_summary}
\end{figure}

\subsection{Distance}

\begin{figure}[h]
    \centering
    \missingfigure{2x2 Subplots: Row 1: Actual vs Predicted scatter with regression, Row 2: SHAP beeswarm top features.}
    \caption{Comparison of performance and top features for distance.}
    \label{fig:obs_distance_summary}
\end{figure}

\subsection{Precipitation}

\begin{figure}[h]
    \centering
    \missingfigure{2x2 Subplots: Row 1: Actual vs Predicted scatter with regression, Row 2: SHAP beeswarm top features.}
    \caption{Comparison of performance and top features for precipitation.}
    \label{fig:obs_precipitation_summary}
\end{figure}