\chapter{Introduction}
\label{ch:intro}

\section{Mesoscale Convective Systems (MSCs) in the Horn of Africa}

\acrfullpl{mcs} have been shown to account for over 60\% of extreme rainfall in Ethiopia and Somalia \citep{Hill2023}. \explainindetail{actual human effects here, deaths, damages, etc} 

Such large-scale storms often lead to intense flooding, which not only can affect the local population, but also countries as far as Egypt due to the Ethiopian Highlands being a watershed origin for countless rivers including the Blue Nile \citep{Legese2020,Zaroug2014}. 

Thus, the importance of understanding these storms, predicting their behaviour, and anticipating their effects cannot be overstated. 

% The Horn of Africa is a region with unique large-scale climatic variability and complex topography that affect \acrshort{mcs} development.

In contrast to Sub-Saharan West Africa, the \acrfull{itcz} passes over the Horn of Africa twice per year providing two distinct rainy seasons, one short and one long \citep{Palmer2023,Tefera2025}. The \acrfull{mjo} is a dominant intra-seasonal factor which modulates rainfall in the tropics and in East Africa, its active phases coincide with increased convection and extreme rainfall events \citep{Camberlin2019,Ochieng2023,Pohl2006}.  The \acrfull{tej} is another key feature enhancing vertical wind shear which can both facilitate or hinder \acrshort{mcs} development \citep{Farnsworth2011,Vashisht2021}. Additionally, the \acrfull{enso} and the \acrfull{iod} have also been shown to modulate rainfall patterns in the region, with coupled regional climate models able to reproduce these patterns at various timescales \citep{Dubache2019,Endris2019,Vashisht2021,Zaroug2014}. 

From the west, the Sahel fades into the Ethiopian Highlands and later the East African Rift Valley, characterised by high topographic relief and complex orography. In the east, the Ethiopian Highlands transition into the low-lying coastal plains of Somalia. Unlike most other countries at this latitude, most of Somalia is arid or semi-arid, with the exception of its border region with Kenya \citep{Beck2023}. This contrast in geography is reflected in storm development and precipitation patterns. The mountains of Ethiopia dominate local convective processes \citep{Negash2024} while the low-lying areas on the south-eastern coast of the region are not nearly as conducive to \acrshort{mcs} development and thus are much more susceptible to storm patterns over the Indian Ocean and the Gulf of Aden \citep{Camberlin2024}. The combination of land surface temperature and soil moisture also impact the storm development and intensification. Notably, multiple studies over distinct regions have demonstrated that strong soil moisture gradients can intensify convection \citep{Barton2021,Klein2020,Taylor2017}. These processes are likewise relevant to the Horn of Africa, especially in transitional climates bridging arid and humid zones. Large-scale \glspl{teleconnection} also play a major role in governing \acrshort{mcs} activity in this region. 

% new paragraph
The primary challenge in this regard is the traditional \acrfull{nwp} models do not resolve convective processes on a scale small enough to accurately predict the behaviour of \acrshortpl{mcs}. \explainindetail{biases over land, at night vs day, underestimate precipitation}

% new paragraph?
In addition, the entire continent of Africa is considerably underrepresented in high-quality meteorological data and many national meteorological services lack critical resources to more effectively monitor and predict \acrshortpl{mcs} \citep{Dinku2019,Kinyondo2018,Meque2021}. This discrepancy also has downstream effects on global meteorology, as reanalysis datasets and digital twin systems rely heavily on data assimilation of observations and regional models for calibration and parametrisation \citep{Linsenmeier2023,Valmassoi2023}. Although the primary remedy would be increased investment and study, 

this discrepancy presents an opportunity for otherwise unorthodox methods to make use of existing data to improve predictions and understanding of \acrshortpl{mcs}.

\section{Explainable Artificial Intelligence for Convective Systems}

% one paragraph
Main surveys:
\begin{itemize}
    \item \href{https://www.sciencedirect.com/science/article/pii/S1352231024004722}{Interpretable machine learning for weather and climate prediction: A review}
    \item \href{https://link.springer.com/chapter/10.1007/978-3-031-04083-2_16#Sec2}{Explainable Artificial Intelligence in Meteorology and Climate Science: Model Fine-Tuning, Calibrating Trust and Learning New Science}
\end{itemize}
\begin{itemize}
    \item Layer-wise Relevance Propagation (LRP) in Tropical Cyclone Intensity and Size Estimation \insertref{\href{https://www.scopus.com/pages/publications/851092020842}{source}}.
    \item Paleoclimate reconstruction of Indian monsoon: \insertref{\href{https://cp.copernicus.org/articles/21/1/2025/}{source}}.
    \item Predicting rainfall extremes in Africa: \insertref{\href{https://www.sciencedirect.com/science/article/abs/pii/B9780443288678000046}{source}}.
    \begin{itemize}
        \item not explicitly XAI, but they did use interpretable models like xgboost
    \end{itemize}
\end{itemize}

The main inspiration for the methodology employed in this thesis comes from \cite{Hunt2024}'s investigation of interpretable gradient-boosted decision-trees for uncovering dynamical relationships governing \gls{indianmonsoon} \acrfull{lps}. They first utilise a tracking algorithm to assemble a database of \acrshort{lps} storm tracks from \acrshort{era5} data. Using the storm centroids along each track, they extract a set of meteorological features from the surrounding \acrshort{era5} data as well as invariant but meaningful features like orographic elevation under the centroid. These features are then used to train a gradient-boosted decision tree model to predict specific storm characteristics, like mean precipitation or peak vorticity. The model is then interpreted using \acrshort{shap} values to not only identify the most important features, but also reveal how feature importance changes relative to geography. The authors posit several notable results including the observation that processes affecting intensification differ from those affecting peak intensity and also confirm existing research that mid-level winds are relevant for predicting storm speed and direction. This methodology is particularly compelling as it also offers a framework that could be adapted for other regions and storm types with the only pre-requisite being high quality storm track data. Such data does already exist in some cases (e.g. the European Windstorm Catalogue \citep{Roberts2014}), but \cite{Hunt2024} show that, for certain storm types, the tracks can be derived from reanalysis data.

However, there are also many additional avenues for exploration, especially leveraging the explanatory power of \acrshort{shap} values. For example, \cite{Hunt2024} primarily focus on the model's ability to capture the relationships between features and storm characteristics, but do not explore the model's ability to predict storm tracks or propagation. While interpretable \acrshort{ml} techniques pale in comparison to traditional \acrfull{nwp} models, their insights may be able to aid with sub-grid parametrisation or facilitate the planning of new observational networks in data-scarce and computationally limited regions like Africa. Furthermore, while \cite{Hunt2024} do take great care to justify the inclusion of each feature and remove those that are highly correlated, they do not explore the possibility of direct comparison between models trained with different feature sets. This could be useful in disentangling the relative importance of meteorological and geographical features, which may often work in concert or direct opposition, such as wind angle and orography. Thus, this thesis aims to build upon their work by exploring these additional avenues and further investigating the potential of interpretable \acrshort{ml} techniques for understanding and predicting storm behaviour.

\section{Research Objectives}

The primary objectives of this research are to investigate the factors which contribute to the intensification and propagation of East African \acrfullpl{mcs} through the usage of an interpretable machine learning model and subsequent application of explainable AI techniques. 

- first, data preprocessing
- model training
- then, explainable AI techniques

\section{Report Structure}

The main matter of this report is organized into 6 chapters with accompanying appendices which contain supplementary material including the full codebase and additional figures.

Chapter~\ref{ch:background} MCSs in africa, data-driven scientific discovery, explainable AI, and review of current applications of XAI in meteorology. 

Chapter~\ref{ch:data} overview of East African MCS database and ERA5 data. 

Chapter~\ref{ch:method} data preprocessing, feature selection, and model training. 

Chapter~\ref{ch:results}  experimental design and subsequent results

Chapter~\ref{ch:discuss} implications of the findings, limitations, assumptions, and potential for future research

Chapter~\ref{ch:con} concludes the report and summarizes the key contributions

