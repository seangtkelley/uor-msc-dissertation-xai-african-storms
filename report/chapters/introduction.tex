\chapter{Introduction}
\label{ch:intro}

\acrfullpl{mcs} have been shown to account for over 60\% of extreme rainfall in Ethiopia and Somalia \citep{Hill2023}. \explainindetail{actual human effects here, deaths, damages, etc} 

Such large-scale storms often lead to intense flooding, which not only can affect the local population, but also countries as far as Egypt due to the Ethiopian Highlands being a watershed origin for countless rivers including the Blue Nile \citep{Legese2020,Zaroug2014}. 

Thus, the importance of understanding these storms, predicting their behaviour, and anticipating their effects cannot be overstated. 

% The Horn of Africa is a region with unique large-scale climatic variability and complex topography that affect \acrshort{mcs} development.

In contrast to Sub-Saharan West Africa, the \acrfull{itcz} passes over the Horn of Africa twice per year providing two distinct rainy seasons, one short and one long \citep{Palmer2023,Tefera2025}. The \acrfull{mjo} is a dominant intra-seasonal factor which modulates rainfall in the tropics and in East Africa, its active phases coincide with increased convection and extreme rainfall events \citep{Camberlin2019,Ochieng2023,Pohl2006}.  The \acrfull{tej} is another key feature enhancing vertical wind shear which can both facilitate or hinder \acrshort{mcs} development \citep{Farnsworth2011,Vashisht2021}. Additionally, the \acrfull{enso} and the \acrfull{iod} have also been shown to modulate rainfall patterns in the region, with coupled regional climate models able to reproduce these patterns at various timescales \citep{Dubache2019,Endris2019,Vashisht2021,Zaroug2014}. 

From the west, the Sahel fades into the Ethiopian Highlands and later the East African Rift Valley, characterised by high topographic relief and complex orography. In the east, the Ethiopian Highlands transition into the low-lying coastal plains of Somalia. Unlike most other countries at this latitude, most of Somalia is arid or semi-arid, with the exception of its border region with Kenya \citep{Beck2023}. This contrast in geography is reflected in storm development and precipitation patterns. The mountains of Ethiopia dominate local convective processes \citep{Negash2024} while the low-lying areas on the south-eastern coast of the region are not nearly as conducive to \acrshort{mcs} development and thus are much more susceptible to storm patterns over the Indian Ocean and the Gulf of Aden \citep{Camberlin2024}. The combination of land surface temperature and soil moisture also impact the storm development and intensification. Notably, multiple studies over distinct regions have demonstrated that strong soil moisture gradients can intensify convection \citep{Barton2021,Klein2020,Taylor2017}. These processes are likewise relevant to the Horn of Africa, especially in transitional climates bridging arid and humid zones. Large-scale \glspl{teleconnection} also play a major role in governing \acrshort{mcs} activity in this region. 

% new paragraph
The primary challenge in this regard is the traditional \acrfull{nwp} models do not resolve convective processes on a scale small enough to accurately predict the behaviour of \acrshortpl{mcs}. \explainindetail{biases over land, at night vs day, underestimate precipitation}

% new paragraph?
In addition, the entire continent of Africa is considerably underrepresented in high-quality meteorological data and many national meteorological services lack critical resources to more effectively monitor and predict \acrshortpl{mcs} \citep{Dinku2019,Kinyondo2018,Meque2021}. This discrepancy also has downstream effects on global meteorology, as reanalysis datasets and digital twin systems rely heavily on data assimilation of observations and regional models for calibration and parametrisation \citep{Linsenmeier2023,Valmassoi2023}. Although the primary remedy would be increased investment and study, 

this discrepancy presents an opportunity for otherwise unorthodox methods to make use of existing data to improve predictions and understanding of \acrshortpl{mcs}.

\section{Research Objectives}

The primary objectives of this research are to investigate the factors which contribute to the intensification and propagation of East African \acrfullpl{mcs} through the usage of an interpretable machine learning model and subsequent application of explainable AI techniques. 

- first, data preprocessing
- model training
- then, explainable AI techniques

\section{Report Structure}

The main matter of this report is organized into 6 chapters with accompanying appendices which contain supplementary material including the full codebase and additional figures.

Chapter~\ref{ch:background} MCSs in africa, data-driven scientific discovery, explainable AI, and review of current applications of XAI in meteorology. 

Chapter~\ref{ch:data} overview of East African MCS database and ERA5 data. 

Chapter~\ref{ch:method} data preprocessing, feature selection, and model training. 

Chapter~\ref{ch:results}  experimental design and subsequent results

Chapter~\ref{ch:discuss} implications of the findings, limitations, assumptions, and potential for future research

Chapter~\ref{ch:con} concludes the report and summarizes the key contributions

